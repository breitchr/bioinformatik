\documentclass[11pt]{article}

\include{Header}

\begin{document}

\Submittedby{Breitenberger, Nicodemus}
\date{\today}

\section{Exercise: Rotation (Ball)}

\begin{lstlisting}
#include <BALL/FORMAT/PDBFile.h>
#include <BALL/KERNEL/system.h>
#include <BALL/KERNEL/chain.h>
#include <BALL/KERNEL/residue.h>
#include <BALL/KERNEL/atom.h>
#include <BALL/KERNEL/protein.h>
#include <BALL/MATHS/angle.h>

int main(int argc, char* argv[]){

   BALL::Vector3 xAxisVector(1,0,0);
   BALL::Angle angle = BALL::Angle(BALL::Constants::PI/3, true);
   BALL::System translationSystem;
   BALL::PDBFile sourceFile;
   BALL::PDBFile rotatedFile;
   BALL::Matrix4x4 rotationMatrix;
   rotationMatrix.setRotation(angle, xAxisVector);

   //Sanity checks for command-line arguments
   if(argc == 3){

      sourceFile.open(argv[1], std::ios::in);
      rotatedFile.open(argv[2], std::ios::out);

   }else{

      std::cout << "Wrong amount of Parameters\n\n Useage: prog inFile outFile\n";
      return 1;

   }

   if(sourceFile.is_open()){

      //the read action for a pdbfile reads the data into a system which can be manipulated afterwards - and saved
      sourceFile.selectModel(1);
      sourceFile.read(translationSystem);
      sourceFile.close();

      // iterate over all proteins
      for(BALL::MoleculeIterator m_it = translationSystem.beginMolecule(); +m_it; ++m_it){

         //We need to check if the molecule we just grabbed with m_it is indeed a protein. if it is not,
         //We cannot iterate over chains and residues.
         if (BALL::RTTI::isKindOf<BALL::Protein>(*(m_it))){

             BALL::Protein* protein = BALL::RTTI::castTo<BALL::Protein>(*(m_it));

             if(protein->countChains() > 0){

               for(BALL::ChainIterator ch_it = protein->beginChain(); +ch_it; ++ch_it){

                  for(BALL::ResidueIterator r_it = ch_it->beginResidue(); +r_it; ++r_it){

                     for(BALL::AtomIterator a_it = r_it->beginAtom(); +a_it; ++a_it){

                        //Rotating Atom-Positions; it may be necessary to move them to origin, rotate and put them back
                        //but since are single Atoms it should not matter
                        a_it->setPosition(rotationMatrix * a_it->getPosition());

                     }                   
                  }
               }
            }
         }
      }

      //writing to out-file
      if(rotatedFile.is_open()){

         rotatedFile.write(translationSystem);

      }else{

            std::cout << "Could not write to system" << '\n';
            return 1;

      }

   }else{

      std::cout << "Could not open PDB file." << '\n';
      return 1;

   }
}
\end{lstlisting}

\section{Exercise: Creation of a molecule (Ball)}

\begin{lstlisting}
#include <BALL/KERNEL/atom.h>
#include <BALL/KERNEL/bond.h>
#include <BALL/KERNEL/PTE.h>
#include <BALL/KERNEL/molecule.h>
#include <BALL/MATHS/vector3.h>
#include <BALL/MATHS/angle.h>
#include <BALL/FORMAT/PDBFile.h>

int main(int argc, char* argv[]){

	BALL::PDBFile outFile;

	//Sanity checks for command-line arguments	
	if(argc == 2){

      outFile.open(argv[1], std::ios::out);

	}else{

    	std::cout << "Wrong amount of Parameters\n\n Useage: prog outFile\n";
    	return 1;

   }

    BALL::Angle phi, theta, phi2, phi3;
    BALL::Atom* nitrogen = new BALL::Atom;//(BALL::PTE[Element::OXYGEN]);
    BALL::Atom* hydro1 = new BALL::Atom;
    BALL::Atom* hydro2 = new BALL::Atom;
    BALL::Atom* hydro3 = new BALL::Atom;

	phi = BALL::Angle(2*M_PI/3, true); //to ensure that the molecule is planar
	theta = BALL::Angle(0, true); //angle to
	phi2 = BALL::Angle(4*M_PI/3, true); //to ensure that the molecule is planar
	phi3 = BALL::Angle(2*M_PI/3, true); //to ensure that the molecule is planar

	BALL::Vector3 nitroPos(0, 0, 0); // nitrogen placed in origin
	BALL::Vector3 hdyro1_1Pos(101.7/100, phi, theta); //101.7/100 = distance in angstrom
	BALL::Vector3 hdyro2_1Pos(101.7/100, theta, phi2);
	BALL::Vector3 hdyro3_1Pos(101.7/100, theta, phi3);

	nitrogen->setElement(BALL::PTE[BALL::Element::NITROGEN]);
	nitrogen->setPosition(nitroPos);
	hydro1->setElement(BALL::PTE[BALL::Element::HYDROGEN]);
	hydro1->setPosition(hdyro1_1Pos);
	hydro2->setElement(BALL::PTE[BALL::Element::HYDROGEN]);
	hydro2->setPosition(hdyro2_1Pos);
	hydro3->setElement(BALL::PTE[BALL::Element::HYDROGEN]);
	hydro3->setPosition(hdyro3_1Pos);

	BALL::Molecule* h3n = new BALL::Molecule;
	BALL::Bond* n_f = hydro1->createBond(*nitrogen);
	BALL::Bond* n_s = hydro2->createBond(*nitrogen);
	BALL::Bond* n_t = hydro3->createBond(*nitrogen);
	
	h3n->append(*nitrogen);
	h3n->append(*hydro1);
	h3n->append(*hydro2);
	h3n->append(*hydro3);

	BALL::System systemHN3("H3N");

	systemHN3.append(*h3n);

	if(outFile.is_open()){

		outFile.write(systemHN3);

	}else{

		std::cout << "Could not write to file" << argv[2] << '\n';

	}	

	return 0;
}
\end{lstlisting}

\section{Exercise: The $\nabla$ operator}

\[ \vec{F} = [3x + y]\vec{i} + [z + 3x]\vec{j} + [x + y + z]\vec{k} \]
Divergence of vector $\vec{F}$:
\[ \nabla \cdot \vec{F} = \partial_x F_x + \partial_y F_y + \partial_z F_z \]
\[ =  \partial_x [3x + y] + \partial_y  [z + 3x] + \partial_z [x + y + z] \]
\[ = 3 + 0 + 1 = 4 \]
Rotation of vector $\vec{F}$:
\[ \nabla \times \vec{F} = \begin{pmatrix}
\partial_y F_z - \partial_z F_y \\
\partial_z F_x - \partial_x F_z \\
\partial_x F_y - \partial_y F_x
\end{pmatrix}  = \begin{pmatrix}
1 - 1 \\
0 - 1 \\
3 - 1
\end{pmatrix} = \begin{pmatrix}
0 \\
-1 \\
2
\end{pmatrix}\]



\[ V = \frac{1}{(x^2 + y^2 + z^2)^3} - \frac{1}{(x^2 + y^2 + z^2)^3}\]
\[ \Rightarrow V = 0 \]
Gradient of scalar V:

\[ \nabla V = 0 \]
Divergence of a scalar is not defined $\nabla \cdot V$ as well as the rotation $\nabla \times V$.

\section{Exercise: Calculation of a potential}

\[ \vec{F} = \vec{- \nabla V} \]
\[ \begin{pmatrix}
F_x \\ F_y \\ F_z
\end{pmatrix}= - 
\begin{pmatrix}
\partial_x V \\
\partial_y V \\
\partial_z V
\end{pmatrix}\]

\[ \vec{F} = 
[6x - y \sin(xy)]\vec{i} + 
[z - x\sin(xy)]\vec{j} + 
\left[y+\frac{1}{1+z^2}\right]\vec{k} \]

\[ F_x = 6x - y \sin(xy) \]
\[ F_y = z - x\sin(xy) \]
\[ F_z = y+\frac{1}{1+z^2} \]


\[ V = - \int F_x  \, \mathrm{d}x + C(y,z) \]
\[ =  - \int 6x - y \sin(xy)  \, \mathrm{d}x + C(y,z) \]
\[ = - (3x^2 + \cos(xy)) + C'(y,z)\]

\[ = -3x^2 - \cos(xy) - \int  F_y  \, \mathrm{d}y + C''(z) \]
\[ =  -3x^2 - \cos(xy) - yz + C''(z) \]
\[ V = -3x^2 - \cos(xy) - yz - \arctan(z) + c \]

with c a constant according to x,y, and z
\end{document}
