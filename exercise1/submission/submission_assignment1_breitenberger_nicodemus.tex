\documentclass[11pt]{article}

\include{Header}

\begin{document}

\Submittedby{Breitenberger, Nicodemus}
\date{\today}

\section{Exercise 1}

\begin{enumerate}
 \item FORMUL record, colum 19 (* for water)
 \item 77-78
 \item x:31-38, y:39-46, z:47-54
 \item 18-20
 \item 22 (chainID)
\end{enumerate}

\section{Exercise 2}

\begin{lstlisting}
#include <iostream>
#include <fstream>
#include <sstream>
#include <vector>

int main(int argc, char* argv[]) {

  std::string line;
  std::ifstream pdbfile; //(argv[1], std::ios::in);
  std::ofstream coordfile; //(argv[2], std::ios::out);

  //Sanity check for correct number of arguments

  switch(argc){

    case 2:
      pdbfile.open(argv[1], std::ios::in);
      break;

    case 3:
      pdbfile.open(argv[1], std::ios::in);
      coordfile.open(argv[2], std::ios::out);
      break;

    default:
      std::cout << "Wrong number of arguments.\n\n Useage: ParsePDB inFile [outFile]" << "\n";
      return 1;
  }
    
  if (pdbfile.is_open()){ 
  
    while (getline(pdbfile,line)){
    
      if (line.substr(0,4) == "ATOM"){

        //x:31-38, y:39-46, z:47-54; one less because arrays in C++ are 0-based
        //a single space is added because it may be, that all coordinates are the maximum allowed size.
        //no trimming is done. could be added later on.
        switch(argc){
          case 2:
            std::cout << line.substr(30,8) << ' ' << line.substr(38,8) << ' ' << line.substr(46,8) << '\n';//substr(position, length); just a reminder for myself
            break;
          case 3:
            coordfile << line.substr(30,8) << ' ' << line.substr(38,8) << ' ' << line.substr(46,8) << '\n';
            break;

        }
	
      }
    }
    pdbfile.close();
    coordfile.close();
  }

  else std::cout << "Unable to open file"; 

  return 0;
}

\end{lstlisting}

\section{Exercise 3}

\begin{lstlisting}
#include <iostream>
#include <fstream>
#include <sstream>
#include <vector>
#include <string>

int main(int argc, char* argv[]) {

  std::string line;
  std::ifstream pdbfile;
  std::ofstream translatefile;
  std::vector<double> d;

  if(argc == 3){

    pdbfile.open(argv[1], std::ios::in);
    translatefile.open(argv[2], std::ios::out);
    d.assign(3,1.000);

  }else if(argc == 6){

    pdbfile.open(argv[1], std::ios::in);
    translatefile.open(argv[2], std::ios::out);
  
    d.push_back(std::stod(argv[3]));
    d.push_back(std::stod(argv[4]));
    d.push_back(std::stod(argv[5]));

  }else{

    std::cout << "Wrong amount of Parameters\n\n Useage: HandlingPDB inFile outFile\n";
    return 1;

  }

  //Basic idea:
  // Checking ATOM
  // Grabbing Coordinates based on columns
  // Parsing to double, adding offset
  // And the c++ way to parsing double to string with fixed size and precision is to set up an outsream correctly
  // Important: If you want a precision of 3, i.e 3 digits after the dot, you need an outstream.precision of 4. Otherwise
  // it get rounded to early.
  // Lastly, replacing the new string in the selected line
  if (pdbfile.is_open()){
    while (getline(pdbfile,line)){
      
      if (line.substr(0,4) == "ATOM"){

        double xCoord = stod(line.substr(30,8));
        double yCoord = stod(line.substr(38,8));
        double zCoord = stod(line.substr(46,8));


        xCoord += d[0];
        yCoord += d[1];
        zCoord += d[2];

        std::ostringstream xCoordStrs;
        std::ostringstream yCoordStrs;
        std::ostringstream zCoordStrs;

        xCoordStrs.width(8);
        xCoordStrs.precision(4);
        yCoordStrs.width(8);
        yCoordStrs.precision(4);
        zCoordStrs.width(8);
        zCoordStrs.precision(4);

        xCoordStrs << xCoord;
        yCoordStrs << yCoord;
        zCoordStrs << zCoord;

        line.replace(30,8,xCoordStrs.str());
        line.replace(38,8,yCoordStrs.str());
        line.replace(46,8,zCoordStrs.str());
      }

      translatefile << line << '\n';

    }

  }else{

    std::cout << "Unable to open file";

  }

  return 0;

}

\end{lstlisting}

\end{document}
