\documentclass[11pt]{article}

\usepackage{palatino}


%\definecolor{links}{HTML}{FF0000}
%\hypersetup{colorlinks,linkcolor=,urlcolor=links}
\usepackage{url}
\usepackage{graphics}
\usepackage{tikz}
\usepackage{hyperref}
\tikzset{every overlay node/.style={draw=black,fill=white,rounded corners,anchor=north west,},}
\def\tikzoverlay{\tikz[baseline,overlay]\node[every overlay node]}

\usepackage{mathtools}
\usepackage{amsfonts}
\usepackage{listings} % Package to include code
\usepackage{color}    

\definecolor{mygreen}{rgb}{0,0.6,0}
\definecolor{mygray}{rgb}{0.5,0.5,0.5}
\definecolor{mymauve}{rgb}{0.58,0,0.82}

\lstset{ %
  backgroundcolor=\color{white},   % choose the background color; you must add \usepackage{color} or \usepackage{xcolor}
  basicstyle=\footnotesize,        % the size of the fonts that are used for the code
  breakatwhitespace=false,         % sets if automatic breaks should only happen at whitespace
  breaklines=true,                 % sets automatic line breaking
  captionpos=b,                    % sets the caption-position to bottom
  commentstyle=\color{mygreen},    % comment style
  deletekeywords={...},            % if you want to delete keywords from the given language
  escapeinside={\%*}{*)},          % if you want to add LaTeX within your code
  extendedchars=true,              % lets you use non-ASCII characters; for 8-bits encodings only, does not work with UTF-8
  frame=single,	                   % adds a frame around the code
  keepspaces=true,                 % keeps spaces in text, useful for keeping indentation of code (possibly needs columns=flexible)
  keywordstyle=\color{blue},       % keyword style
  language=C++,                    % the language of the code
  otherkeywords={*,...},           % if you want to add more keywords to the set
  numbers=left,                    % where to put the line-numbers; possible values are (none, left, right)
  numbersep=5pt,                   % how far the line-numbers are from the code
  numberstyle=\tiny\color{mygray}, % the style that is used for the line-numbers
  rulecolor=\color{black},         % if not set, the frame-color may be changed on line-breaks within not-black text (e.g. comments (green here))
  showspaces=false,                % show spaces everywhere adding particular underscores; it overrides 'showstringspaces'
  showstringspaces=false,          % underline spaces within strings only
  showtabs=false,                  % show tabs within strings adding particular underscores
  stepnumber=1,                    % the step between two line-numbers. If it's 1, each line will be numbered
  stringstyle=\color{mymauve},     % string literal style
  tabsize=2,	                    % sets default tabsize to 2 spaces
  title=\lstname                   % show the filename of files included with \lstinputlisting; also try caption instead of title
}


\newcommand{\Submittedby}[1]
{
	\begin{center}
	\scshape \large
	Submitted by: #1 \\
	\end{center}
}



\begin{document}

\Submittedby{Breitenberger, Nicodemus}
\date{\today}

\section{Exercise 1}

\begin{enumerate}
 \item FORMUL record, colum 19 (* for water)
 \item 77-78
 \item x:31-38, y:39-46, z:47-54
 \item 18-20
 \item 22 (chainID)
\end{enumerate}

\section{Exercise 2}

\begin{lstlisting}
#include <iostream>
#include <fstream>
#include <sstream>
#include <vector>

int main(int argc, char* argv[]) {

  std::string line;
  std::ifstream pdbfile; //(argv[1], std::ios::in);
  std::ofstream coordfile; //(argv[2], std::ios::out);

  //Sanity check for correct number of arguments

  switch(argc){

    case 2:
      pdbfile.open(argv[1], std::ios::in);
      break;

    case 3:
      pdbfile.open(argv[1], std::ios::in);
      coordfile.open(argv[2], std::ios::out);
      break;

    default:
      std::cout << "Wrong number of arguments.\n\n Useage: ParsePDB inFile [outFile]" << "\n";
      return 1;
  }
    
  if (pdbfile.is_open()){ 
  
    while (getline(pdbfile,line)){
    
      if (line.substr(0,4) == "ATOM"){

        //x:31-38, y:39-46, z:47-54; one less because arrays in C++ are 0-based
        //a single space is added because it may be, that all coordinates are the maximum allowed size.
        //no trimming is done. could be added later on.
        switch(argc){
          case 2:
            std::cout << line.substr(30,8) << ' ' << line.substr(38,8) << ' ' << line.substr(46,8) << '\n';//substr(position, length); just a reminder for myself
            break;
          case 3:
            coordfile << line.substr(30,8) << ' ' << line.substr(38,8) << ' ' << line.substr(46,8) << '\n';
            break;

        }
	
      }
    }
    pdbfile.close();
    coordfile.close();
  }

  else std::cout << "Unable to open file"; 

  return 0;
}

\end{lstlisting}

\section{Exercise 3}

\begin{lstlisting}
#include <iostream>
#include <fstream>
#include <sstream>
#include <vector>
#include <string>

int main(int argc, char* argv[]) {

  std::string line;
  std::ifstream pdbfile;
  std::ofstream translatefile;
  std::vector<double> d;

  if(argc == 3){

    pdbfile.open(argv[1], std::ios::in);
    translatefile.open(argv[2], std::ios::out);
    d.assign(3,1.000);

  }else if(argc == 6){

    pdbfile.open(argv[1], std::ios::in);
    translatefile.open(argv[2], std::ios::out);
  
    d.push_back(std::stod(argv[3]));
    d.push_back(std::stod(argv[4]));
    d.push_back(std::stod(argv[5]));

  }else{

    std::cout << "Wrong amount of Parameters\n\n Useage: HandlingPDB inFile outFile\n";
    return 1;

  }

  //Basic idea:
  // Checking ATOM
  // Grabbing Coordinates based on columns
  // Parsing to double, adding offset
  // And the c++ way to parsing double to string with fixed size and precision is to set up an outsream correctly
  // Important: If you want a precision of 3, i.e 3 digits after the dot, you need an outstream.precision of 4. Otherwise
  // it get rounded to early.
  // Lastly, replacing the new string in the selected line
  if (pdbfile.is_open()){
    while (getline(pdbfile,line)){
      
      if (line.substr(0,4) == "ATOM"){

        double xCoord = stod(line.substr(30,8));
        double yCoord = stod(line.substr(38,8));
        double zCoord = stod(line.substr(46,8));


        xCoord += d[0];
        yCoord += d[1];
        zCoord += d[2];

        std::ostringstream xCoordStrs;
        std::ostringstream yCoordStrs;
        std::ostringstream zCoordStrs;

        xCoordStrs.width(8);
        xCoordStrs.precision(4);
        yCoordStrs.width(8);
        yCoordStrs.precision(4);
        zCoordStrs.width(8);
        zCoordStrs.precision(4);

        xCoordStrs << xCoord;
        yCoordStrs << yCoord;
        zCoordStrs << zCoord;

        line.replace(30,8,xCoordStrs.str());
        line.replace(38,8,yCoordStrs.str());
        line.replace(46,8,zCoordStrs.str());
      }

      translatefile << line << '\n';

    }

  }else{

    std::cout << "Unable to open file";

  }

  return 0;

}

\end{lstlisting}

\end{document}
