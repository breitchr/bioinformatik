\documentclass[11pt]{article}

\include{Header}

\begin{document}

\Submittedby{Breitenberger, Nicodemus}
\date{\today}

\section{Exercise 1}

\begin{lstlisting}

#include <BALL/FORMAT/PDBFile.h>
#include <BALL/KERNEL/system.h>
#include <BALL/KERNEL/chain.h>
#include <BALL/KERNEL/residue.h>
#include <BALL/KERNEL/atom.h>
#include <BALL/KERNEL/protein.h>


int main(int argc, char* argv[]){

   BALL::Vector3 translationVector(1,1,1);
   BALL::System translationSystem;
   BALL::PDBFile sourceFile;
   BALL::PDBFile translateFile;

   //Sanity checks for command-line arguments
   if(argc == 3){

      sourceFile.open(argv[1], std::ios::in);
      translateFile.open(argv[2], std::ios::out);

   }else if(argc == 6){

   sourceFile.open(argv[1], std::ios::in);
   translateFile.open(argv[2], std::ios::out);
  
   translationVector = BALL::Vector3(std::stod(argv[3]), std::stod(argv[4]), std::stod(argv[5]));

   }else{

      std::cout << "Wrong amount of Parameters\n\n Useage: prog inFile outFile\n";
      return 1;

   }

   if(sourceFile.is_open()){

      //the read action for a pdbfile reads the data into a system which can be manipulated afterwards - and saved
      sourceFile.read(translationSystem);
      sourceFile.close();

      // iterate over all proteins
      for(BALL::MoleculeIterator m_it = translationSystem.beginMolecule(); +m_it; ++m_it){

         //We need to check if the molecule we just grabbed with m_it is indeed a protein. if it is not,
         //We cannot iterate over chains and residues.
         if (BALL::RTTI::isKindOf<BALL::Protein>(*(m_it))){

             BALL::Protein* protein = BALL::RTTI::castTo<BALL::Protein>(*(m_it));

             if(protein->countChains() > 0){

               for(BALL::ChainIterator ch_it = protein->beginChain(); +ch_it; ++ch_it){

                  for(BALL::ResidueIterator r_it = ch_it->beginResidue(); +r_it; ++r_it){

                     for(BALL::AtomIterator a_it = r_it->beginAtom(); +a_it; ++a_it){

                        //Exercise 1: Translating Atom-Positions
                        a_it->setPosition(a_it->getPosition() + translationVector);
                     }                   
                  }
               }
            }
         }
      }
   }else{

      std::cout << "Could not open PDB file." << '\n';

   }

   //Writing translatedFile


   if(translateFile.is_open()){

      translateFile.write(translationSystem);
      translateFile.close();

   }else{

      std::cout << "Could not write translated PDB file." << '\n';

   }

   return 0;

}

\end{lstlisting}

\section{Exercise 2}

\begin{lstlisting}

#include <BALL/FORMAT/PDBFile.h>
#include <BALL/KERNEL/system.h>
#include <BALL/KERNEL/chain.h>
#include <BALL/KERNEL/residue.h>
#include <BALL/KERNEL/atom.h>
#include <BALL/KERNEL/protein.h>


int main(int argc, char* argv[]){

   BALL::Vector3 translationVector(1,1,1);
   BALL::System translationSystem;
   BALL::PDBFile sourceFile;
   BALL::PDBFile translateFile;

   //Sanity checks for command-line arguments
   if(argc == 2){

      sourceFile.open(argv[1], std::ios::in);

   }else{

      std::cout << "Wrong amount of Parameters\n\n Useage: prog inFile\n";
      return 1;

   }

   if(sourceFile.is_open()){

      //the read action for a pdbfile reads the data into a system which can be manipulated afterwards - and saved
      sourceFile.read(translationSystem);
      sourceFile.close();

      // iterate over all proteins
      for(BALL::MoleculeIterator m_it = translationSystem.beginMolecule(); +m_it; ++m_it){

         //We need to check if the molecule we just grabbed with m_it is indeed a protein. if it is not,
         //We cannot iterate over chains and residues.
         if (BALL::RTTI::isKindOf<BALL::Protein>(*(m_it))){

             BALL::Protein* protein = BALL::RTTI::castTo<BALL::Protein>(*(m_it));

             if(protein->countChains() > 0){

               for(BALL::ChainIterator ch_it = protein->beginChain(); +ch_it; ++ch_it){

                  for(BALL::ResidueIterator r_it = ch_it->beginResidue(); +r_it; ++r_it){

                     for(BALL::AtomIterator a_it = r_it->beginAtom(); +a_it; ++a_it){

                        //Exercise 2: Printing C-Alphas
                        if(a_it->getName() == "CA"){

                           std::cout << a_it->getPosition() << '\n';

                        }
                     }                   
                  }
               }
            }
         }
      }
   }else{

      std::cout << "Could not open PDB file." << '\n';

   }

   return 0;

}

\end{lstlisting}

\section{Exercise 3}

\subsection{$ y''(x) + y(x) = \sin(2x) $}

General solution will be from the form $y(x) = y_c(x) + y_p(x)$, which is the sum of the complementary and particular solution.

First determine the complementary solution: 
\[ y''(x) + y(x) = 0 \]

Which can be solved by a sin(x) as well as a cos(x). Because 

\[  \partial^2_x \sin(x) = -\sin(x)  \]
\[  \partial^2_x \cos(x) = -\cos(x)  \]

Therefore

\[ y_c(x) = A\sin(x) + B\cos(x) \]

Second step is to determine the particular solution:

\[ y_p(x) = C\sin(2x) + D\cos(2x) \]
\[ \partial^2_x y_p(x) = -4C\sin(2x) -4D\cos(2x) \]

\[ -4C\sin(2x) -4D\cos(2x) + C\sin(2x) + D\cos(2x) = \sin(2x) \]
\[ -3C\sin(2x) - 3D\cos(2x) = \sin(2x) \]
\[ \Rightarrow D = 0 ~~~~ C = -\frac{1}{3}\sin(2x) \]

\[\boxed{ y(x) = A\sin(x) + B\cos(x) -\frac{1}{3}\sin(2x) }\]
Proof:
\[ y''(x) = -A\sin(x) - B\cos(x) -\frac{4}{3} \sin(2x)\]
\[ y''(x) + y(x) = \sin(x) + B\cos(x) -\frac{1}{3}\sin(2x) -A\sin(x) - B\cos(x) -\frac{4}{3} \sin(2x) = \sin(2x) \]

\subsection{$ y''(x) + y(x) = \sin(x) $}

Complementary solution same as in section 3.1.

\[ y_c(x) = A\sin(x) + B\cos(x) \]

Guess for particular solution:
\[ y_p(x) = C x \cos(x)\]
\[ y'_p(x) = C \cos(x) - C x \sin(x) \]
\[ y''_p(x) = - C \sin(x) - C x \cos(x) - C \sin(x) = -2 C \sin(x) - C x \cos(x) \]
\[ -2 C \sin(x) - C x \cos(x) + C x \cos(x) = \sin(x) \]
\[ \Rightarrow C = -\frac{1}{2} \]

\[\boxed{ y(x) = A\sin(x) + B\cos(x) -\frac{1}{2}x\cos(x) }\]

Proof:
\[ y''(x) = - A\sin(x) - B\cos(x) + \sin(x) - \frac{1}{2} x \cos(x) \]
\[ y''(x) + y(x) = A\sin(x) + B\cos(x) -\frac{1}{2}x\cos(x) - A\sin(x) - B\cos(x) + \sin(x) - \frac{1}{2} x \cos(x) \]
\[ = \sin(x) \]

\subsection{$ y''(x) - y(x) = 0 $}

This is a homogeneous differential equation. Therefore it is only necessary to find the homogeneous solution $y_c(x)$.

The equation can be solved by $e^x$ and $e^-x$. 

\[\boxed{ y(x) = Ae^x + Be^{-x} }\]  
Proof:

\[ y''(x) = Ae^x + (-1)^2Be^{-x} = Ae^x + Be^{-x}\]

\[ y''(x) - y(x) = Ae^x + Be^{-x} - Ae^x - Be^{-x} = 0\]
\end{document}
