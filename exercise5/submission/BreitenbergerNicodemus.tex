\documentclass[11pt]{article}

\usepackage{palatino}


%\definecolor{links}{HTML}{FF0000}
%\hypersetup{colorlinks,linkcolor=,urlcolor=links}
\usepackage{url}
\usepackage{graphics}
\usepackage{tikz}
\usepackage{hyperref}
\tikzset{every overlay node/.style={draw=black,fill=white,rounded corners,anchor=north west,},}
\def\tikzoverlay{\tikz[baseline,overlay]\node[every overlay node]}

\usepackage{mathtools}
\usepackage{amsfonts}
\usepackage{listings} % Package to include code
\usepackage{color}    

\definecolor{mygreen}{rgb}{0,0.6,0}
\definecolor{mygray}{rgb}{0.5,0.5,0.5}
\definecolor{mymauve}{rgb}{0.58,0,0.82}

\lstset{ %
  backgroundcolor=\color{white},   % choose the background color; you must add \usepackage{color} or \usepackage{xcolor}
  basicstyle=\footnotesize,        % the size of the fonts that are used for the code
  breakatwhitespace=false,         % sets if automatic breaks should only happen at whitespace
  breaklines=true,                 % sets automatic line breaking
  captionpos=b,                    % sets the caption-position to bottom
  commentstyle=\color{mygreen},    % comment style
  deletekeywords={...},            % if you want to delete keywords from the given language
  escapeinside={\%*}{*)},          % if you want to add LaTeX within your code
  extendedchars=true,              % lets you use non-ASCII characters; for 8-bits encodings only, does not work with UTF-8
  frame=single,	                   % adds a frame around the code
  keepspaces=true,                 % keeps spaces in text, useful for keeping indentation of code (possibly needs columns=flexible)
  keywordstyle=\color{blue},       % keyword style
  language=C++,                    % the language of the code
  otherkeywords={*,...},           % if you want to add more keywords to the set
  numbers=left,                    % where to put the line-numbers; possible values are (none, left, right)
  numbersep=5pt,                   % how far the line-numbers are from the code
  numberstyle=\tiny\color{mygray}, % the style that is used for the line-numbers
  rulecolor=\color{black},         % if not set, the frame-color may be changed on line-breaks within not-black text (e.g. comments (green here))
  showspaces=false,                % show spaces everywhere adding particular underscores; it overrides 'showstringspaces'
  showstringspaces=false,          % underline spaces within strings only
  showtabs=false,                  % show tabs within strings adding particular underscores
  stepnumber=1,                    % the step between two line-numbers. If it's 1, each line will be numbered
  stringstyle=\color{mymauve},     % string literal style
  tabsize=2,	                    % sets default tabsize to 2 spaces
  title=\lstname                   % show the filename of files included with \lstinputlisting; also try caption instead of title
}


\newcommand{\Submittedby}[1]
{
	\begin{center}
	\scshape \large
	Submitted by: #1 \\
	\end{center}
}



\begin{document}

\Submittedby{Breitenberger, Nicodemus}
\date{\today}

\section{Exercise: Simple coarse grained model}

\section{Exercise: Nos\`{e}-Hoover}
\begin{lstlisting}
#include <iostream>
#include <fstream>
#include <cmath>

using namespace std;

//v is fixed, calculation of v(t+1/2dt) not necessary

double v(double t, double dt, double fixed){
   if (t <= 0){
      return 1.0;
   }
   else {
      // fix velocity to 0.8 or 1.2
      return fixed;  //1/2*(v(t-1/2*dt,dt)+v(t+1/2*dt,dt));
   }
}

// x(t) for a harmonic oszillator with m=k=1
double x(double t){
   double x0 = 1.0;

   if(t == 0){
      return x0;
   }
   else {
      return (x0*cos(t));
   }
}

double x(double t, double dt, double fixedv){
   return (x(t) + dt*v(t,dt,fixedv));
}

int main(int argc, char* argv[]) {

   ofstream outfile1;
   ofstream outfile2;
   outfile1.open("NoseHooverOut08.txt");
   outfile2.open("NoseHooverOut12.txt");

   outfile1 << "t x(t) v(t)" << endl;
   outfile2 << "t x(t) v(t)" << endl;
   double dt = 0.1;

   for (double t = 0; t < stod(argv[1]); t+=dt){
      outfile1 << t << "  " << x(t, dt, 0.8) << "  " << v(t, dt, 0.8) << endl;
      outfile2 << t << "  " << x(t, dt, 1.2) << "  " << v(t, dt, 1.2) << endl;
   }

   outfile1.close();
   outfile2.close();
   
   // plot x(t) and v(t)
   system("gnuplot plot.gnu ");

   return 0;
}

\end{lstlisting}
\newpage
\begin{lstlisting}
# gnuplot script to plot Nose-Hoover output
set   autoscale                        # scale axes automatically
unset log                              # remove any log-scaling
unset label                            # remove any previous labels
set xtic auto                          # set xtics automatically
set ytic auto                          # set ytics automatically
set title "Simulation of a harmonic oszillator with fixed velocity"
set xlabel "t"
set ylabel "x(t), v(t)"
set xr [0.0:100]
set yr [-2:2]


set size 2,2
set origin 0,0
set multiplot layout 2,1 columnsfirst scale 1,1

plot "NoseHooverOut08.txt" using 1:3 title 'v(t) fixed to 0.8' with linespoints, "NoseHooverOut08.txt" using 1:2 title 'x(t)' with linespoints

plot "NoseHooverOut12.txt" using 1:3 title 'v(t) fixed to 1.2' with linespoints, "NoseHooverOut12.txt" using 1:2 title 'x(t)' with linespoints

unset multiplot	
\end{lstlisting}

\newpage
\section{Exercise: Molecular Dynamics - BALL}

\begin{lstlisting}
#include <BALL/KERNEL/system.h>
#include <BALL/KERNEL/selector.h>
#include <BALL/FORMAT/PDBFile.h>
#include <BALL/MOLMEC/MDSIMULATION/microCanonicalMD.h>
#include <BALL/STRUCTURE/fragmentDB.h>
#include <BALL/STRUCTURE/residueChecker.h>
#include <BALL/MOLMEC/AMBER/amber.h>
#include <BALL/MOLMEC/MINIMIZATION/conjugateGradient.h>

using namespace std;
using namespace BALL;

int main(int argc, char* argv[]){

   PDBFile sourceFile;
   System mdSystem;

   //Sanity checks for command-line arguments	
   if(argc == 3){
      sourceFile.open(argv[1], ios::in);
   }else{
      cout << "Wrong amount of Parameters\n\n Useage: prog sourceFile simulationTime\n";
      return 1;
   }

   if(sourceFile.is_open()){
      sourceFile.read(mdSystem);
      sourceFile.close();
   }

   FragmentDB db("");
   mdSystem.apply(db.normalize_names);
   mdSystem.apply(db.add_hydrogens);
   mdSystem.apply(db.build_bonds);

   ResidueChecker rc(db);
   mdSystem.apply(rc);

   // create hydrogen bonds and force field 
   AmberFF amber(mdSystem);
   Selector hydrogen_selector("element(H)");
   mdSystem.apply(hydrogen_selector);

   amber.options[PeriodicBoundary::Option::PERIODIC_BOX_ENABLED]="true";
   amber.setup(mdSystem);

   MicroCanonicalMD md(amber);
   md.setReferenceTemperature(300.0);
   md.setEnergyOutputFrequency(500.0);

   // redirect std::cout to file. Found no other possibility to write MD simulation output directly to file 
   ofstream finalMD("finalMD.txt");
   streambuf *coutbuf = cout.rdbuf();
   cout.rdbuf(finalMD.rdbuf());
   md.simulateTime(stod(argv[2]));
   cout.rdbuf(coutbuf);
   finalMD.close();
   return 0;
}

\end{lstlisting}

\end{document}
