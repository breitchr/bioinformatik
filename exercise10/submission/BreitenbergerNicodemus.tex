\documentclass[11pt]{article}
\usepackage{bbold}
\usepackage{palatino}


%\definecolor{links}{HTML}{FF0000}
%\hypersetup{colorlinks,linkcolor=,urlcolor=links}
\usepackage{url}
\usepackage{graphics}
\usepackage{tikz}
\usepackage{hyperref}
\tikzset{every overlay node/.style={draw=black,fill=white,rounded corners,anchor=north west,},}
\def\tikzoverlay{\tikz[baseline,overlay]\node[every overlay node]}

\usepackage{mathtools}
\usepackage{amsfonts}
\usepackage{listings} % Package to include code
\usepackage{color}    

\definecolor{mygreen}{rgb}{0,0.6,0}
\definecolor{mygray}{rgb}{0.5,0.5,0.5}
\definecolor{mymauve}{rgb}{0.58,0,0.82}

\lstset{ %
  backgroundcolor=\color{white},   % choose the background color; you must add \usepackage{color} or \usepackage{xcolor}
  basicstyle=\footnotesize,        % the size of the fonts that are used for the code
  breakatwhitespace=false,         % sets if automatic breaks should only happen at whitespace
  breaklines=true,                 % sets automatic line breaking
  captionpos=b,                    % sets the caption-position to bottom
  commentstyle=\color{mygreen},    % comment style
  deletekeywords={...},            % if you want to delete keywords from the given language
  escapeinside={\%*}{*)},          % if you want to add LaTeX within your code
  extendedchars=true,              % lets you use non-ASCII characters; for 8-bits encodings only, does not work with UTF-8
  frame=single,	                   % adds a frame around the code
  keepspaces=true,                 % keeps spaces in text, useful for keeping indentation of code (possibly needs columns=flexible)
  keywordstyle=\color{blue},       % keyword style
  language=C++,                    % the language of the code
  otherkeywords={*,...},           % if you want to add more keywords to the set
  numbers=left,                    % where to put the line-numbers; possible values are (none, left, right)
  numbersep=5pt,                   % how far the line-numbers are from the code
  numberstyle=\tiny\color{mygray}, % the style that is used for the line-numbers
  rulecolor=\color{black},         % if not set, the frame-color may be changed on line-breaks within not-black text (e.g. comments (green here))
  showspaces=false,                % show spaces everywhere adding particular underscores; it overrides 'showstringspaces'
  showstringspaces=false,          % underline spaces within strings only
  showtabs=false,                  % show tabs within strings adding particular underscores
  stepnumber=1,                    % the step between two line-numbers. If it's 1, each line will be numbered
  stringstyle=\color{mymauve},     % string literal style
  tabsize=2,	                    % sets default tabsize to 2 spaces
  title=\lstname                   % show the filename of files included with \lstinputlisting; also try caption instead of title
}


\newcommand{\Submittedby}[1]
{
	\begin{center}
	\scshape \large
	Submitted by: #1 \\
	\end{center}
}



\begin{document}

\Submittedby{Breitenberger, Nicodemus}
\date{\today}

\section{Protein Docking}
\begin{lstlisting}


//Basic idea: dock one monomer after the other, buiding a longer chain with each docking.
//During each docking, check for clashes and pass the N (in the example its 4) possible
//candidates with the lowest energy without clashes to the next docking. This ensures
//that we get a wider array of possible solutions

define N 4 //number of possible candidates to pass on
define SIZE 5 //number of wanted size of monomer-1

ListOfDockedProtein dock2Proteins(base, add){
  ListOfCandidates = execute rosetta base add;
  
  ListOfEnergies = [];
  for candidate in ListOfCandidates{
  
    calculate energy;
    check for clash;
    if(clash in candidate){
    
      set energy of candidate to +infinity;
      
    }else{
    
      set energy of candidate;
    
    }
  
  }
  
  ListOfDockedProtein[N];
  for energy in ListOfEnergies{
  
    get N lowest energies of all possible energies;
  
  }
  
  return ListOfDockedProtein;
  
int main(arguments){

  read monomer from file;
  
  ListOfDockedProtein Current = dock2Proteins(monomer, monomer);
  
  counter = 0;
  repeat until counter = SIZE{
  
    ListOfListOfDockedProteins = [];
  
    i = 0;
    for each protein in Current{
    
      ListOfListOfDockedProteins[i] = dock2Proteins(protein, monomer);
      i++;
      
    }
    
    ListOfDockedProtein Temp = [];
    for each Protein in ListOfListOfDockedProteins{
      
      check energies;
      enter the four lowest energies into Temp;
	
     }
    
    Current = Temp;
    counter++;
  
  }
  
  //Do a last energy check
  for each protein in Current{
  
    check energies;
    return protein with lowest energy;
  
  }
  
}

\end{lstlisting}

\section{Score Calculation}
\begin{lstlisting}

\end{lstlisting}


\end{document}
