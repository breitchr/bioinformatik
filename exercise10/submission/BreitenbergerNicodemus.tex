\documentclass[11pt]{article}
\usepackage{bbold}
\include{Header}

\begin{document}

\Submittedby{Breitenberger, Nicodemus}
\date{\today}

\section{Protein Docking}
\begin{lstlisting}


//Basic idea: dock one monomer after the other, buiding a longer chain with each docking.
//During each docking, check for clashes and pass the N (in the example its 4) possible
//candidates with the lowest energy without clashes to the next docking. This ensures
//that we get a wider array of possible solutions

define N 4 //number of possible candidates to pass on
define SIZE 5 //number of wanted size of monomer-1

ListOfDockedProtein dock2Proteins(base, add){
  ListOfCandidates = execute rosetta base add;
  
  ListOfEnergies = [];
  for candidate in ListOfCandidates{
  
    calculate energy;
    check for clash;
    if(clash in candidate){
    
      set energy of candidate to +infinity;
      
    }else{
    
      set energy of candidate;
    
    }
  
  }
  
  ListOfDockedProtein[N];
  for energy in ListOfEnergies{
  
    get N lowest energies of all possible energies;
  
  }
  
  return ListOfDockedProtein;
  
int main(arguments){

  read monomer from file;
  
  ListOfDockedProtein Current = dock2Proteins(monomer, monomer);
  
  counter = 0;
  repeat until counter = SIZE{
  
    ListOfListOfDockedProteins = [];
  
    i = 0;
    for each protein in Current{
    
      ListOfListOfDockedProteins[i] = dock2Proteins(protein, monomer);
      i++;
      
    }
    
    ListOfDockedProtein Temp = [];
    for each Protein in ListOfListOfDockedProteins{
      
      check energies;
      enter the four lowest energies into Temp;
	
     }
    
    Current = Temp;
    counter++;
  
  }
  
  //Do a last energy check
  for each protein in Current{
  
    check energies;
    return protein with lowest energy;
  
  }
  
}

\end{lstlisting}

\section{Score Calculation}
\begin{lstlisting}

\end{lstlisting}


\end{document}
