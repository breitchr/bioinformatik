\documentclass[11pt]{article}

\usepackage{palatino}


%\definecolor{links}{HTML}{FF0000}
%\hypersetup{colorlinks,linkcolor=,urlcolor=links}
\usepackage{url}
\usepackage{graphics}
\usepackage{tikz}
\usepackage{hyperref}
\tikzset{every overlay node/.style={draw=black,fill=white,rounded corners,anchor=north west,},}
\def\tikzoverlay{\tikz[baseline,overlay]\node[every overlay node]}

\usepackage{mathtools}
\usepackage{amsfonts}
\usepackage{listings} % Package to include code
\usepackage{color}    

\definecolor{mygreen}{rgb}{0,0.6,0}
\definecolor{mygray}{rgb}{0.5,0.5,0.5}
\definecolor{mymauve}{rgb}{0.58,0,0.82}

\lstset{ %
  backgroundcolor=\color{white},   % choose the background color; you must add \usepackage{color} or \usepackage{xcolor}
  basicstyle=\footnotesize,        % the size of the fonts that are used for the code
  breakatwhitespace=false,         % sets if automatic breaks should only happen at whitespace
  breaklines=true,                 % sets automatic line breaking
  captionpos=b,                    % sets the caption-position to bottom
  commentstyle=\color{mygreen},    % comment style
  deletekeywords={...},            % if you want to delete keywords from the given language
  escapeinside={\%*}{*)},          % if you want to add LaTeX within your code
  extendedchars=true,              % lets you use non-ASCII characters; for 8-bits encodings only, does not work with UTF-8
  frame=single,	                   % adds a frame around the code
  keepspaces=true,                 % keeps spaces in text, useful for keeping indentation of code (possibly needs columns=flexible)
  keywordstyle=\color{blue},       % keyword style
  language=C++,                    % the language of the code
  otherkeywords={*,...},           % if you want to add more keywords to the set
  numbers=left,                    % where to put the line-numbers; possible values are (none, left, right)
  numbersep=5pt,                   % how far the line-numbers are from the code
  numberstyle=\tiny\color{mygray}, % the style that is used for the line-numbers
  rulecolor=\color{black},         % if not set, the frame-color may be changed on line-breaks within not-black text (e.g. comments (green here))
  showspaces=false,                % show spaces everywhere adding particular underscores; it overrides 'showstringspaces'
  showstringspaces=false,          % underline spaces within strings only
  showtabs=false,                  % show tabs within strings adding particular underscores
  stepnumber=1,                    % the step between two line-numbers. If it's 1, each line will be numbered
  stringstyle=\color{mymauve},     % string literal style
  tabsize=2,	                    % sets default tabsize to 2 spaces
  title=\lstname                   % show the filename of files included with \lstinputlisting; also try caption instead of title
}


\newcommand{\Submittedby}[1]
{
	\begin{center}
	\scshape \large
	Submitted by: #1 \\
	\end{center}
}



\begin{document}

\Submittedby{Breitenberger, Nicodemus}
\date{\today}

\section{Exercise: Calculation of the work}
Point moving in force field:

\[ \vec{F} = (3x + y)\vec{i} + (x + 2y)\vec{j} \]

Two paths:

\[ \gamma_1(t) = \begin{cases}
                 t \vec{i} & t \in [0,3) \\
                 3 \vec{i} + (t-3) \vec{j} & t \in [3,6] \\
                 \end{cases}
 \]
\[ \gamma_2(t) = \begin{cases}
                 t \vec{i} + t \vec{j} & t \in [0,3) \\
                 \end{cases}
\]

Calculate work :

\[ W_{ab} = \int_{\vec{r}_a}^{\vec{r}_b} \vec{F}(\vec{r}) d\vec{r}  = \int_{t_a}^{t_b} \vec{F}(\vec{r}) \frac{d\vec{r}}{dt} dt \]


With $ \vec{r}(t) = \gamma_1(t) $:

\[ W_{ab} = \int_{0}^{3} ((3t + 0)\vec{i} + (t + 0)\vec{j}) \frac{d (t \vec{i})}{dt} dt + \int_{3}^{6} ((9 + (t-3))\vec{i} + (3 + 2(t-3))\vec{j}) \frac{d (3 \vec{i} + (t-3)\vec{j})}{dt} dt \]
\[ W_{ab} = \int_{0}^{3} (9 + (t-3)) dt +  \int_{3}^{6} (3 + 2(t-3)) dt \]
\[ = \int_{0}^{3} (6 + t) dt +  \int_{3}^{6} (2t -3) dt = \left[ 6t + \frac{1}{2}t^2\right]_0^3 + \left[ t^2 - 3t\right]_3^6 \]
\[ = 18 + 4.5 + 36 - 18 - 9 + 9 = 40.5 \]

With $ \vec{r}(t) = \gamma_2(t) $:

\[ W_{ab} = \int_{0}^{3} ((3t + t)\vec{i} + (t + 2t)\vec{j}) \frac{d(t \vec{i} + t \vec{j})}{dt} dt \]
\[ = \int_{0}^{3} ((3t + t)\vec{i} + (t + 2t)\vec{j}) (\vec{i} + \vec{j}) dt \]
\[ = \int_{0}^{3} (3t + t + t + 2t) dt = \int_{0}^{3} (3t + t + t + 2t) dt = \int_{0}^{3} 7t dt \]
\[ = \left[ \frac{7}{2} t^2 \right]_0^3 = \frac{7}{2}9 = 31.5 \]

The work depends on the path the point goes through the force field. 

\section{Exercise: Generating a grid for Katchalski-Katzir}

\begin{lstlisting}
#include <BALL/FORMAT/PDBFile.h>
#include <BALL/KERNEL/system.h>
#include <BALL/KERNEL/chain.h>
#include <BALL/KERNEL/residue.h>
#include <BALL/KERNEL/atom.h>
#include <BALL/KERNEL/protein.h>
#include <BALL/KERNEL/PTE.h>
#include <BALL/MATHS/vector3.h>
#include <array>
#include <iostream>
#include <fstream>

using namespace std;
using namespace BALL;

int alpha1 = 1;
int beta1 = 0;
int gamma1 = 2;

int main(int argc, char* argv[]){

  //   BALL::Vector3 translationVector(1,1,1);
  System kkSystem;
  PDBFile sourceFile;
  // BALL::PDBFile translateFile;

  vector<Vector3> atompos;
  Vector3 pos;
  vector<float> vwradii;
  float vwr;

  //Sanity checks for command-line arguments
  if(argc == 2){

    sourceFile.open(argv[1], ios::in);
  }else{

    cout << "Wrong amount of Parameters\n\n Useage: prog inFile \n";
    return 1;

  }

  if(sourceFile.is_open()){

    //the read action for a pdbfile reads the data into a system which can be manipulated afterwards - and saved
    sourceFile.read(kkSystem);
    sourceFile.close();

    // get only first protein
    Protein* protein = kkSystem.getProtein(0);

    if(protein->countChains() > 0){

      for(ChainIterator ch_it = protein->beginChain(); +ch_it; ++ch_it){

        for(ResidueIterator r_it = ch_it->beginResidue(); +r_it; ++r_it){

          for(AtomIterator a_it = r_it->beginAtom(); +a_it; ++a_it){
            Element element = a_it->getElement();
            // name
            cout << element.getName() << " ";
            // position
            pos = a_it->getPosition();
            // save position in vector
            atompos.push_back(pos);
            cout << pos  << " ";
            // van der waals radius
            vwr = element.getVanDerWaalsRadius();
            cout << vwr << endl;
            vwradii.push_back(vwr);
          }                   
        }
      }
    }
    for(int i(0); i < atompos.size(); i++){
      cout << atompos[i] << vwradii[i] << endl;
    }
  }else{
    cout << "Could not open PDB file." << '\n';
  }

  // dimension of katchalski katzir grid n*m*l
  int n = 50;
  int m = 50;
  int l = 50;
  float px;
  float py;
  float pz;

  float kkgrid[n][m][l];
  // initialize kkgrid
  for (int i = 0; i < n; i++){

    for (int j = 0; j < m; j++){

      for (int k = 0; k < l; k++){
        int c = 0;
        while (c < atompos.size()){

          px = 0.1*(i-n/2);
          py = 0.1*(j-m/2);
          pz = 0.1*(k-l/2);

          // inside 
          if (px > (atompos[c][0] - 0.1*vwradii[c]) && px < (atompos[c][0] + 0.1*vwradii[c]) && py > (atompos[c][1] - 0.1*vwradii[c]) && py < (atompos[c][1] + 0.1*vwradii[c]) && pz > (atompos[c][2] - 0.1*vwradii[c]) && pz < (atompos[c][2] + 0.1*vwradii[c]) ){
            kkgrid[i][j][k] = beta1;
          }
          // outside
          else if  (px < (atompos[c][0] - 0.1*vwradii[c]) ||  px > (atompos[c][0] + 0.1*vwradii[c]) || py > (atompos[c][1] - 0.1*vwradii[c]) || py < (atompos[c][1] + 0.1*vwradii[c]) || pz > (atompos[c][2] - 0.1*vwradii[c]) || pz < (atompos[c][2] + 0.1*vwradii[c]) ){
            kkgrid[i][j][k] = gamma1;
          }
          // boundary
          else{
            kkgrid[i][j][k] = alpha1;
          }
          c++;
        }
      }  
    }
 }

  // safe kkgrid to file
  ofstream kkfile;
  kkfile.open("kkgrid.txt");
  kkfile << "x y z value" << endl;

  for (int x(0); x < n; x++){

    for (int y(0); y < m; y++){

      for (int z(0); z < l; z++){

        kkfile << 0.1*(x-n/2) << " " << 0.1*(y-m/2) << " " << 0.1*(z-l/2) << " " << kkgrid[x][y][z] << endl; 
      }
    }
  }

  kkfile.close();

  // countour plots
  system("gnuplot plotkkgrid.gnu");

  return 0;

}

\end{lstlisting}


Gnuplot script "plotkkgrid.gnu":

\begin{lstlisting}
# gnuplot script to plot katchalski katzir output

set parametric 
set contour base 
set view 0,0,1
unset surface
set cntrparam levels 5
set dgrid3d

set title "Katchalski-Katzir Grid entries for different planes"

set size 2,2
set origin 0,0
set multiplot layout 3,1 columnsfirst scale 1,1

set xr [-10.0:10.0]
set yr [-10.0:10.0]
set zr [0:2]
set xlabel "x"
set ylabel "y"
splot "kkgrid.txt" using 1:2:4 title "x-y plane" with line

set xr [-10.0:10.0]
set yr [-10.0:10.0]
set zr [0:2]
set xlabel "y"
set ylabel "z"
plot "kkgrid.txt" using 2:3:4 title "y-z plane" with line

set xr [-10.0:10.0]
set yr [-10.0:10.0]
set zr [0:2]
set xlabel "x"
set ylabel "z"
plot "kkgrid.txt" using 1:3:4 title "x-z plane" with line

unset multiplot
\end{lstlisting}

\end{document}
