\documentclass[11pt]{article}
\usepackage{bbold}
\usepackage{palatino}


%\definecolor{links}{HTML}{FF0000}
%\hypersetup{colorlinks,linkcolor=,urlcolor=links}
\usepackage{url}
\usepackage{graphics}
\usepackage{tikz}
\usepackage{hyperref}
\tikzset{every overlay node/.style={draw=black,fill=white,rounded corners,anchor=north west,},}
\def\tikzoverlay{\tikz[baseline,overlay]\node[every overlay node]}

\usepackage{mathtools}
\usepackage{amsfonts}
\usepackage{listings} % Package to include code
\usepackage{color}    

\definecolor{mygreen}{rgb}{0,0.6,0}
\definecolor{mygray}{rgb}{0.5,0.5,0.5}
\definecolor{mymauve}{rgb}{0.58,0,0.82}

\lstset{ %
  backgroundcolor=\color{white},   % choose the background color; you must add \usepackage{color} or \usepackage{xcolor}
  basicstyle=\footnotesize,        % the size of the fonts that are used for the code
  breakatwhitespace=false,         % sets if automatic breaks should only happen at whitespace
  breaklines=true,                 % sets automatic line breaking
  captionpos=b,                    % sets the caption-position to bottom
  commentstyle=\color{mygreen},    % comment style
  deletekeywords={...},            % if you want to delete keywords from the given language
  escapeinside={\%*}{*)},          % if you want to add LaTeX within your code
  extendedchars=true,              % lets you use non-ASCII characters; for 8-bits encodings only, does not work with UTF-8
  frame=single,	                   % adds a frame around the code
  keepspaces=true,                 % keeps spaces in text, useful for keeping indentation of code (possibly needs columns=flexible)
  keywordstyle=\color{blue},       % keyword style
  language=C++,                    % the language of the code
  otherkeywords={*,...},           % if you want to add more keywords to the set
  numbers=left,                    % where to put the line-numbers; possible values are (none, left, right)
  numbersep=5pt,                   % how far the line-numbers are from the code
  numberstyle=\tiny\color{mygray}, % the style that is used for the line-numbers
  rulecolor=\color{black},         % if not set, the frame-color may be changed on line-breaks within not-black text (e.g. comments (green here))
  showspaces=false,                % show spaces everywhere adding particular underscores; it overrides 'showstringspaces'
  showstringspaces=false,          % underline spaces within strings only
  showtabs=false,                  % show tabs within strings adding particular underscores
  stepnumber=1,                    % the step between two line-numbers. If it's 1, each line will be numbered
  stringstyle=\color{mymauve},     % string literal style
  tabsize=2,	                    % sets default tabsize to 2 spaces
  title=\lstname                   % show the filename of files included with \lstinputlisting; also try caption instead of title
}


\newcommand{\Submittedby}[1]
{
	\begin{center}
	\scshape \large
	Submitted by: #1 \\
	\end{center}
}



\begin{document}

\Submittedby{Breitenberger, Nicodemus}
\date{\today}

\section{Exercise: Minimize RMSD}

\[ \textrm{RMSD}(A,B) = \min_{R,T} \sqrt{\frac{1}{N}\sum_i || A_i - R (B_i - T) ||^2 }\]

No rotation $\Rightarrow$ R is equal to identity matrix.

\[ \textrm{RMSD}(A,B) = \min_{T} \sqrt{\frac{1}{N}\sum_i || A_i - (B_i - T) ||^2 }\]

Minimize with respect to T $\Rightarrow \partial_T \textrm{RMSD}(A,B) = 0 $

\[ \partial_T \sqrt{\frac{1}{N}\sum_i || A_i - (B_i - T) ||^2} \]
\[ = \partial_T \sqrt{\frac{1}{N} \sum_i \left(\sqrt{\sum_i A_i - (B_i - T) }\right)^2} \]
\[ = \partial_T \sqrt{\frac{1}{N} \sum_i  A_i - (B_i - T)} = 0\]

\[ \Rightarrow T = \frac{1}{N} \sum_i (B_i - A_i) \]

This is equivalent to moving the center of mass of A into the center of mass of B.

\section{Molecular Dynamics -- BALL}
\begin{lstlisting}
 #include <BALL/FORMAT/PDBFile.h>
#include <BALL/STRUCTURE/peptides.h>
#include <BALL/STRUCTURE/peptideBuilder.h>
#include <BALL/STRUCTURE/fragmentDB.h>
#include <BALL/KERNEL/system.h>
#include <BALL/KERNEL/chain.h>
#include <BALL/KERNEL/protein.h>

using namespace std;
using namespace BALL;

int main(int argc, char* argv[]){
  
  PDBFile sourceFile;
  System mdSystem;
  string url = "http://www.rcsb.org/pdb/files/";
  string pdbid;

  //Sanity checks for command-line arguments	
  if(argc == 2){
    pdbid.append(argv[1] + string(".pdb"));
    url.append(pdbid);
  }else{
    cout << "Wrong amount of Parameters\n\n Useage: prog pdbid" << endl;
    return 1;
  }

  // download pdb file (-N : if not exists or newer version online)
  system((string("wget -N ") + url).c_str());
  sourceFile.open(pdbid);
  if(sourceFile.is_open()){
    sourceFile.read(mdSystem);
    sourceFile.close();
  }

  if (mdSystem.getProtein(0)){
    Protein* protein = mdSystem.getProtein(0);
    Chain* chain = protein->getChain(0);
    String sequence = Peptides::GetSequence(*chain);
    // test sequence of aminoacids
    cout << sequence << endl;

    // create new long sequence of aminoacids
    String newSeq;
    for (int i = 0; i < 10; i++){
      newSeq.append(sequence);
    }

    Peptides::PeptideBuilder* pb = new Peptides::PeptideBuilder(newSeq);

    FragmentDB fdb("");
    pb->setFragmentDB(&fdb);
    
    pb->setChainName("new_long_Chain");
    pb->setProteinName("new_Protein");
    
    Protein* newProt = pb->construct();
    
    // test newProt
    cout << Peptides::GetSequence(*newProt) << endl;

    //System newSystem;
    mdSystem.insert(*newProt);
    PDBFile outFile(string("new_")+pdbid,ios::out);
    outFile << mdSystem;
    outFile.close();

  }
 
  return 0;
}

\end{lstlisting}


\end{document}
