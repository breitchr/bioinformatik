\documentclass[11pt]{article}
\usepackage{bbold}
\include{Header}

\begin{document}

\Submittedby{Breitenberger, Nicodemus}
\date{\today}

\section{Exercise: Minimize RMSD}

\[ \textrm{RMSD}(A,B) = \min_{R,T} \sqrt{\frac{1}{N}\sum_i || A_i - R (B_i - T) ||^2 }\]

No rotation $\Rightarrow$ R is equal to identity matrix.

\[ \textrm{RMSD}(A,B) = \min_{T} \sqrt{\frac{1}{N}\sum_i || A_i - (B_i - T) ||^2 }\]

Minimize with respect to T $\Rightarrow \partial_T \textrm{RMSD}(A,B) = 0 $

\[ \partial_T \sqrt{\frac{1}{N}\sum_i || A_i - (B_i - T) ||^2} \]
\[ = \partial_T \sqrt{\frac{1}{N} \sum_i \left(\sqrt{\sum_i A_i - (B_i - T) }\right)^2} \]
\[ = \partial_T \sqrt{\frac{1}{N} \sum_i  A_i - (B_i - T)} = 0\]

\[ \Rightarrow T = \frac{1}{N} \sum_i (B_i - A_i) \]

This is equivalent to moving the center of mass of A into the center of mass of B.

\section{Molecular Dynamics -- BALL}
\begin{lstlisting}
 #include <BALL/FORMAT/PDBFile.h>
#include <BALL/STRUCTURE/peptides.h>
#include <BALL/STRUCTURE/peptideBuilder.h>
#include <BALL/STRUCTURE/fragmentDB.h>
#include <BALL/KERNEL/system.h>
#include <BALL/KERNEL/chain.h>
#include <BALL/KERNEL/protein.h>

using namespace std;
using namespace BALL;

int main(int argc, char* argv[]){
  
  PDBFile sourceFile;
  System mdSystem;
  string url = "http://www.rcsb.org/pdb/files/";
  string pdbid;

  //Sanity checks for command-line arguments	
  if(argc == 2){
    pdbid.append(argv[1] + string(".pdb"));
    url.append(pdbid);
  }else{
    cout << "Wrong amount of Parameters\n\n Useage: prog pdbid" << endl;
    return 1;
  }

  // download pdb file (-N : if not exists or newer version online)
  system((string("wget -N ") + url).c_str());
  sourceFile.open(pdbid);
  if(sourceFile.is_open()){
    sourceFile.read(mdSystem);
    sourceFile.close();
  }

  if (mdSystem.getProtein(0)){
    Protein* protein = mdSystem.getProtein(0);
    Chain* chain = protein->getChain(0);
    String sequence = Peptides::GetSequence(*chain);
    // test sequence of aminoacids
    cout << sequence << endl;

    // create new long sequence of aminoacids
    String newSeq;
    for (int i = 0; i < 10; i++){
      newSeq.append(sequence);
    }

    Peptides::PeptideBuilder* pb = new Peptides::PeptideBuilder(newSeq);

    FragmentDB fdb("");
    pb->setFragmentDB(&fdb);
    
    pb->setChainName("new_long_Chain");
    pb->setProteinName("new_Protein");
    
    Protein* newProt = pb->construct();
    
    // test newProt
    cout << Peptides::GetSequence(*newProt) << endl;

    //System newSystem;
    mdSystem.insert(*newProt);
    PDBFile outFile(string("new_")+pdbid,ios::out);
    outFile << mdSystem;
    outFile.close();

  }
 
  return 0;
}

\end{lstlisting}


\end{document}
