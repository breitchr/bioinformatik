\documentclass[11pt]{article}

\usepackage{palatino}


%\definecolor{links}{HTML}{FF0000}
%\hypersetup{colorlinks,linkcolor=,urlcolor=links}
\usepackage{url}
\usepackage{graphics}
\usepackage{tikz}
\usepackage{hyperref}
\tikzset{every overlay node/.style={draw=black,fill=white,rounded corners,anchor=north west,},}
\def\tikzoverlay{\tikz[baseline,overlay]\node[every overlay node]}

\usepackage{mathtools}
\usepackage{amsfonts}
\usepackage{listings} % Package to include code
\usepackage{color}    

\definecolor{mygreen}{rgb}{0,0.6,0}
\definecolor{mygray}{rgb}{0.5,0.5,0.5}
\definecolor{mymauve}{rgb}{0.58,0,0.82}

\lstset{ %
  backgroundcolor=\color{white},   % choose the background color; you must add \usepackage{color} or \usepackage{xcolor}
  basicstyle=\footnotesize,        % the size of the fonts that are used for the code
  breakatwhitespace=false,         % sets if automatic breaks should only happen at whitespace
  breaklines=true,                 % sets automatic line breaking
  captionpos=b,                    % sets the caption-position to bottom
  commentstyle=\color{mygreen},    % comment style
  deletekeywords={...},            % if you want to delete keywords from the given language
  escapeinside={\%*}{*)},          % if you want to add LaTeX within your code
  extendedchars=true,              % lets you use non-ASCII characters; for 8-bits encodings only, does not work with UTF-8
  frame=single,	                   % adds a frame around the code
  keepspaces=true,                 % keeps spaces in text, useful for keeping indentation of code (possibly needs columns=flexible)
  keywordstyle=\color{blue},       % keyword style
  language=C++,                    % the language of the code
  otherkeywords={*,...},           % if you want to add more keywords to the set
  numbers=left,                    % where to put the line-numbers; possible values are (none, left, right)
  numbersep=5pt,                   % how far the line-numbers are from the code
  numberstyle=\tiny\color{mygray}, % the style that is used for the line-numbers
  rulecolor=\color{black},         % if not set, the frame-color may be changed on line-breaks within not-black text (e.g. comments (green here))
  showspaces=false,                % show spaces everywhere adding particular underscores; it overrides 'showstringspaces'
  showstringspaces=false,          % underline spaces within strings only
  showtabs=false,                  % show tabs within strings adding particular underscores
  stepnumber=1,                    % the step between two line-numbers. If it's 1, each line will be numbered
  stringstyle=\color{mymauve},     % string literal style
  tabsize=2,	                    % sets default tabsize to 2 spaces
  title=\lstname                   % show the filename of files included with \lstinputlisting; also try caption instead of title
}


\newcommand{\Submittedby}[1]
{
	\begin{center}
	\scshape \large
	Submitted by: #1 \\
	\end{center}
}



\begin{document}

\Submittedby{Breitenberger, Nicodemus}
\date{\today}

\section{Exercise: Solving a simple model -- system differential equations}

Consider a water molecule. Assuming that the position of the oxygen atom is fixed and the angle between the hydrogen atom is also fixed the only thing that can change is the distance between the oxygen and hydrogen atoms. These are defined as $\rho_1$ and $\rho_2$. In the next steps described with $\rho_j $, $j \in\{1,2\}$.

The connection of the oxygen und hydrogen atom can be approximated with a spring with a spring constant k. 

Therefore equating Newton's force with the spring force results in:

\[ m \partial^2_t \rho_j = -k\rho_j \]
\[ \partial^2_t \rho_j + \frac{k}{m}\rho_j = 0  \]

with the initial conditions:
\[ \rho_j(0) = \tilde{\rho_j} \]
\[ \partial_t\rho_j(0) = v_j \]

$ \rho_j(t) $ has to fullfill the equation as well as the initial conditions. Assume an exponential function:

\[ \rho_j(t) = c e^{\lambda t} \]

\[ \partial^2_t\rho_j(t) = c \lambda^2 e^{\lambda t} \]

Insert into differential equation:

\[  c \lambda^2 e^{\lambda t} + c \frac{k}{m} e^{\lambda t} = 0 \]
\[ \lambda^2 + \frac{k}{m} = 0 \]
\[ \lambda_{1,2} = \pm\sqrt{-\frac{k}{m}} = \pm i \sqrt{\frac{k}{m}} \]

Add the two complex solutions:

\[ \rho_j(t) = c_1 \cdot e^{i\sqrt{\frac{k}{m}}t} + c_2 \cdot e^{-i\sqrt{\frac{k}{m}}t} \]

Initial conditions:

\[ \rho_j(0) = c_1 + c_2 = \tilde{\rho_j} \]

This is fullfilled for:

\[ \rho_j(t) =\tilde{\rho}_j \cos{\left(\sqrt{\frac{k}{m}}t\right)}\]

with $c_1 = c_2 = \frac{1}{2}\tilde{\rho}_j$

\[ \partial_t \rho_j(t) = - \tilde{\rho}_j \sqrt{\frac{k}{m}} \sin{\left(\sqrt{\frac{k}{m}}t\right)}  \]

\[ \partial_t \rho_j(0) = - \tilde{\rho}_j \sqrt{\frac{k}{m}} \sin{\left(0\right)} = 0 \]

This means for the velocity at t = 0: 

\[ v_j(t=0) = 0 \]

If we approximate the binding between the oxygen and the hydrogen atoms each with a spring force the initial conditions lay down that at the starting point t=0 the distances are at their maximum $\tilde{\rho_j}$. At time t=0 the velocity is zero because the motion is at a turning point. This motion of the water molecule is called the symmetric stretch mode.

\section{Exercise: Lennard--Jones}
 \begin{lstlisting}
 # Gnuplot script file for plotting lennard jones potential and force
 set   autoscale                      # scale axes automatically
 unset log                            # remove any log-scaling
 unset label                          # remove any previous labels
 set xtic auto                        # set xtics automatically
 set ytic auto                        # set ytics automatically
 
 set title "Lennard-Jones potential and approximation"
 set xlabel "r"
 set ylabel "V(r)"
 
 set xrange [11:13]
 set yrange [-3:0]
 
 LJV(x) = 2.0*((12.0/x)**12.0-2.0*(12.0/x)**6.0) 
 V(x) = -2.0 + (6.0*2.0)/(12.0**2.0)*(13.0-7.0)*(x-12.0)**2.0
 
 set multiplot layout 1,2 
 set size 1,0.5
 set origin 0,0
 
 plot V(x) title "Approximation close to equilibrium point", LJV(x) title "Lennard-Jones Potential"
 
 #set term png
 #set output "LennardJones.png"
 #replot
 #set term x11
 
 # o.b.d.A. vector r = |r|*(1,0,0)
 # only in x direction because gnuplot does not provide 
 # a 3d vector plot of a function
 
 LJF(x) = 12.0*2.0/x * ((12.0/x)**12.0 - (12.0/x)**6.0)
 F(x) = -72.0*(x-12.0)/(12.0**2.0)
 
 set title "Lennard-Jones force and approximation"
 set xlabel "x"
 set ylabel "F(x)"
 
 set xrange [10:15]
 set yrange [-20:20]
 
 set size 1,0.5
 set origin 0,0.5
 plot F(x) title "Approximation close to equilibrium point", LJF(x) title "Lennard-Jones Force"
 
 unset multiplot
 
 \end{lstlisting}
 
 The plots verify that the approximation is reasonable close to the equilibrium point. Further away from this point the approximated curves are very different from the Lennard-Jones potential and force.
 \\
 \\
\textbf{Morse potential:}  $ V(r) = \epsilon \left(1-e^{-\sigma(r-r_0)}\right)^2$
 \\
 \\
 Corresponding force:
\[ \vec{F}(\vec{r}) = - \nabla V = 2 \epsilon \sigma e^{-\sigma(r-r_0)} \left(1-e^{-\sigma(r-r_0)}\right) \frac{\vec{r}}{r} \]
  
Approximation: Taylor expansion of the potential as in the exercise description for minimum (equilibrium point $\tilde{r}$).
\[ \frac{\partial V}{\partial r}\bigg|_{r=\tilde{r}} = 0 = 2 \epsilon \sigma e^{-\sigma(\tilde{r}-r_0)} \left(1-e^{-\sigma(\tilde{r}-r_0)}\right) \Leftrightarrow \tilde{r} = r_0 \]

\[ \frac{\partial^2 V}{\partial r^2}\bigg|_{r = \tilde{r}} = 2\epsilon\sigma^2 e^{-2 \sigma (\tilde{r}-r_0)} - 2 \epsilon \sigma^2 e^{-\sigma(\tilde{r}-r_0)} \left(1-e^{-\sigma(\tilde{r}-r_0)}\right) \]

\[ V(r) = V(r_0) + \frac{\partial V(r_0)}{\partial r}(r-r_0) + \frac{1}{2} \frac{\partial^2 V(r_0)}{\partial r^2}(r-r_0)^2 + O((r-r_0)^3) \]

\[ V(r) \approx 0 + 0 + 2 \epsilon \sigma^2 (r-r_0)^2 = 2 \epsilon \sigma^2 (r-r_0)^2 \]

\[ \vec{F}(\vec{r}) = - \nabla V \approx 4 \epsilon \sigma^2 (r-r_0) \frac{\vec{r}}{r} \]


Then the linear approximation constant equals $k_M = 4 \epsilon \sigma^2$.

\section{Exercise: Euler Method -- C++}
\begin{lstlisting}
#include <string>
#include <iostream>
#include <fstream>
#include <sstream>

//could be replaced by literal constants; define is the C way
#define K 1
#define M 1
#define H 0.01
#define STEPS 1000
//some constants regarding x_0 and v_0
#define X_0 0
#define V_0 2


double computeNextForce(double x_n_1){

	return -1 * K * x_n_1;

}

double computeNextVelocity(double v_n_1, double F_n){

	return v_n_1 + H / M * F_n;

}

double computeNextPosition(double x_n_1, double v_n_1){

	return x_n_1 + H * v_n_1;

}

int main(int argc, char* argv[]){

	double x_n_1 = X_0;
	double v_n_1 = V_0;
	double f_n = 0;
	std::ofstream eulerFile, gnuPlot;

	if(argc == 2){

		eulerFile.open(argv[1], std::ios::out);
		gnuPlot.open("gp.gp", std::ios::out);

	}else{

		std::cout << "Wrong number of arguments.\n\n Useage: prog outfile" << '\n';
		return 1;

	}

	int i = 1;
	for(i = 1; i <= STEPS; i++){

		f_n = computeNextForce(x_n_1);
		double v_n_temp = computeNextVelocity(v_n_1, f_n);
		double x_n_temp = computeNextPosition(x_n_1, v_n_1);

		v_n_1 = v_n_temp;
		x_n_1 = x_n_temp;

		eulerFile << i * H << ' ' << f_n  << '\n';
		eulerFile << i * H << ' ' << v_n_1  << '\n';
		eulerFile << i * H << ' ' << x_n_1  << '\n';

	}

	eulerFile.close();

	//preparing a gnu-script file to print the data to a png-file
	//basically, i set a gnuplot file to read the generated file and print it to a png file
	//most of the stuff is hardcoded. There has to be a more elegant way. For now, this will suffice
	gnuPlot << "set terminal png" << '\n';
	gnuPlot << "plot \"" << argv[1] << "\" using 1:2 title \"Euler Method Plot for Force, Velocity and Position\"" << '\n';
	gnuPlot.close();

	std::system("gnuplot gp.gp > eulerMethod.png");


}
\end{lstlisting}
\end{document}
